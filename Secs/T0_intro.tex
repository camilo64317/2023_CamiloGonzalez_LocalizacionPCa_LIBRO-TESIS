% ------------------------------------------------------------------------
% ------------------------------------------------------------------------
% ------------------------------------------------------------------------
%                              Introducción
% ------------------------------------------------------------------------
% ------------------------------------------------------------------------
% ------------------------------------------------------------------------


\nnchapter{INTRODUCCIÓN} %max 5 pags
% ------------------------------------------------------------------------
% ------------------------------------------------------------------------


% 1er parrafo. Impacto 

El cáncer de próstata (PCa) es el segundo cáncer más común en hombres y en la mayoría de los países se establece como el más diagnosticado. En 2020, a nivel global hubo 1.4 millones de casos nuevos y más de 300.000 fallecimientos. Además, se estima que las muertes asociadas a este cáncer se duplicarán para el año 2040 \myfootcite{Sung2021}. A nivel mundial, el 6.8\% de las muertes por cáncer en hombres son por cáncer de próstata, mientras que en Colombia esta cifra ha presentado un incremento constante en los últimos 20 años, siendo actualmente la causante del 14.5\% de las muertes asociadas a cáncer en hombres  \myfootcite{Sung2021,asivamosensalud}.\par

% -- 2do parrafo. Problema y situacion actual (Como lo hacen y problemas...). 
En la práctica clínica, se busca identificar lesiones clínicamente significativas de cáncer de próstata (csPca, por sus siglas en inglés), cuya detección temprana permite reducir el número de lesiones agresivas y la cantidad de muertes asociadas a este cáncer \myfootcite{uro2020014}. Por ello, un diagnóstico preciso y oportuno es relevante para diseñar mejores tratamientos que impacten positivamente en el paciente. A la fecha, los métodos mas utilizados durante la rutina clínica para el diagnóstico inicial, o sospecha de PCa, son el examen de sangre de antígeno prostático (prostate specific antigen, PSA) y el examen rectal digital (digital rectal exam, DRE) \myfootcite{Ftterer2015,Wang2021,Smith2004}. Sin embargo, se ha mostrado que el examen PSA presenta una baja especificidad, alcanzando cifras cercanas al 20\% \myfootcite{Merriel2022}. Por otra parte, el DRE es un método invasivo, altamente dependiente de la experiencia del experto y no permite examinar zonas importantes de la glándula prostática, ignorando así potenciales lesiones csPCa \myfootcite{david2022prostate,Naji2018}.\par


% ------ BPMRI as potential.  
El análisis de secuencias de resonancia magnética multi-paramétrica (MP-MRI) se ha constituido en los últimos años como una alternativa prometedora para soportar el diagnóstico temprano en una fase previa a la biopsia \myfootcite{Deniffel2021}. Estas secuencias se usan como una herramienta estándar para el diagnóstico o el tamizaje (`screening') poblacional en diferentes regiones en el mundo \myfootcite{10.1001/jamaoncol.2020.7456,Saar2020}. En especial, permiten caracterizar lesiones, incluso en regiones alejadas de la pared rectal y zona transicional \myfootcite{murphy2013expanding,Stabile2019}. Sin embargo, el enfoque MP-MRI implica la utilización de agentes de contraste, que pueden tener efectos secundarios como malestar general o irritación en la piel, debido a la inyección del agente de contraste. Así mismo, requiere un tiempo considerable para generar su secuencia de imágenes, y por si fuera poco, ya se ha relatado sobre una posible no evidencia estadística significativa con el uso de su secuencia (DCE) \myfootcite{Behzadi2018,GraciaBara2022,Schoots2021,Tan2015}. De manera que, el uso de secuencias de resonancia magnética bi-paramétricas (bp-MRI) ha emergido como una alternativa, que,  en comparación con las secuencias MP-MRI, resultan hasta tres veces más rápidas, no requieren agentes de contraste y tienen una precisión equiparable en el soporte al diagnóstico \myfootcite{Pecoraro2021,Obmann2018,Alver2022,Steinkohl2018}.
A pesar de ello, el análisis de lesiones en secuencias MRI es afectado por la variabilidad que existe entre lecturas por parte de diferentes radiólogos, subjetividad que puede inducir errores en la caracterización y localización de lesiones csPCa. Particularmente, la caracterización de estas lesiones en secuencias bp-MRI es una tarea compleja para los radiólogos, su competencia se logra con experiencia, tanto en años, como en número de interpretaciones \myfootcite{Salka2022,Kang2021}. Sin embargo, 
no son muchos los radiólogos que cuentan con esta experiencia, y en eventuales diagnósticos masivos, ya existe preocupación por posible escasez de personal \myfootcite{Mata2021,Rimmer2017}. A propósito de lo anterior, se han propuesto ambiciosos programas para fortalecer tareas de tamizaje basados en estudios MRI que permitan mitigar la mortalidad asociada a este cáncer \myfootcite{Bratt2023}. Ahora bien, en relación a resultados coetáneos, de programas de tamizaje previo, se evidencia que las labores de clasificación, o determinación de un grado en una escala para este tipo de cáncer no han sido muy efectivas, pues no hay evidencia de disminución en la mortalidad \myfootcite{Hamdy2023}. Anejando por lo dicho, para una correcta caracterización, resultarían necesarias herramientas que especialicen la tarea de localización de lesiones sobre estas secuencias MRI.\par

%existen metodos computacionales....de clasificacion... pero estos parten de una localización
En los últimos años, el uso de herramientas computacionales ha surgido como una herramienta prometedora para ayudar en el diagnóstico y localización de esta enfermedad \myfootcite{Shah2022,Rouvire2023}. Entre estos métodos, sobresalen los modelos de aprendizaje profundo, los cuales permiten clasificar lesiones clínicamente significativas con una precisión sobresaliente e incluso comparable con la de expertos lectores radiólogos \myfootcite{Mata2021}. A pesar de ello, el diseño de estas estrategias supone un conjunto de datos que incluya una localización precisa de las lesiones para ajustar los modelos \myfootcite{CastilloT2021}. Algunos trabajos relacionados comprenden, la determinación de la agresividad de lesiones csPCA en resonancias, a partir de correlación con imágenes histopatológicas \myfootcite{Seetharaman2021}, predicción de lesiones csPCA utilizando estrategias de tranfer learning \myfootcite{Chen2019}, predicción del grado histopátologico de Gleason sobre secuencias MRI, a partir del uso de clasificadores KNN \myfootcite{Jensen2019}, o la detección y clasificación a través de redes U-NET en cascada \myfootcite{Mehralivandcas2022}. Así pues, la mayoría de estrategias intentan resolver la detección y/o clasificación de lesiones csPCA, más no dilucidan el potencial de poder especializarse en la localización de lesiones potencialmente mortales. Por lo tanto, resulta relevante el diseño de herramientas de aprendizaje automático que resulten precisas y eficaces en tareas de localización de lesiones csPCA.

Este trabajo propone la implementación de una herramienta de aprendizaje profundo para la localización de lesiones csPCA en estudios bp-MRI que sirva de apoyo a labores radiológicas y urológicas de (`screening'). Para ello, se seleccionará un conjunto de datos que contenga imágenes bp-MRI, junto con anotaciones relacionadas con la localización y grado de malignidad de las lesiones. Estos datos serán utilizados para desarrollar y entrenar un algoritmo de aprendizaje profundo con la capacidad de localizar lesiones csPCA. Además, se evaluará la efectividad del algoritmo en comparación con los métodos tradicionales y se discutirán las implicaciones clínicas de su uso. 