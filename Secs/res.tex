% ------------------------------------------------------------------------
% ------------------------------------------------------------------------
% ------------------------------------------------------------------------
%                                Resumen
% ------------------------------------------------------------------------
% ------------------------------------------------------------------------
% ------------------------------------------------------------------------

\chapter*{RESUMEN}

\footnotesize{
\begin{description}
  \item[TÍTULO:] Localización de lesiones relacionadas con el cáncer de próstata sobre secuencias multimodales bp-MRI\astfootnote{Trabajo de investigación}
  \item[AUTOR:] Camilo Eduardo González Guerrero\asttfootnote{Facultad de Ingenierías Fisicomecánicas Escuela de Ingeniería de Sistemas e Informática. Director: Fabio Martínez Carrillo, Ph.D. Codirector: Juan Andrés Olmos Rojas, M.Sc. }
  \item[PALABRAS CLAVE:] Cáncer de próstata, localización, lesión csPCa, bpMRI, multimodal.
  \item[DESCRIPCIÓN:] 
El cáncer de próstata es el segundo cáncer con mayor incidencia en hombres a nivel mundial. En Colombia, por ejemplo, la tasa de defunción es de alrededor de 11.6 casos por cada 100 mil habitantes. Hoy en día, el estudio de lesiones prostáticas mediante resonancia magnética biparamétrica es un criterio estándar para la detección y diagnóstico del cáncer de próstata, incluso en estadios tempranos. Sin embargo, la localización de estas lesiones sigue siendo subjetiva y su caracterización reporta bajos niveles de sensibilidad. Es por ello por lo que los mecanismos computacionales son hoy en día claves para la localización, diagnóstico y triage del cáncer de próstata sobre los estudios bp-MRI. En esta propuesta se pretende desarrollar una herramienta de aprendizaje profundo para localizar lesiones prostáticas a partir de la observación de múltiples parámetros de la resonancia. Para lo anterior, se seleccionará un conjunto de datos debidamente anotado por expertos, se ajustará y adaptará una herramienta de aprendizaje profundo. Se espera validar el esquema propuesto en cuanto a la capacidad de localizar lesiones y discriminarlas según la estratificación del conjunto de datos. 
 
\end{description}
}\normalsize
% ------------------------------------------------------------------------ 