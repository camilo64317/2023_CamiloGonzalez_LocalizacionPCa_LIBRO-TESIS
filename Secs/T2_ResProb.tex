

\chapter{PROBLEMA DE INVESTIGACIÓN}

El cáncer de próstata es una enfermedad que afecta a un gran número de hombres en todo el mundo y su detección temprana es crucial para reducir la agresividad y la cantidad de muertes asociadas. Aunque el estudio de lesiones prostáticas mediante resonancia magnética biparamétrica (bp-MRI) es un criterio estándar para la detección y diagnóstico del cáncer de próstata, la localización de estas lesiones sigue siendo subjetiva a la experticia del radiólogo y su caracterización reporta bajos niveles de sensibilidad. Además, hay diferencias notables entre el diagnóstico por parte de diferentes expertos. Lo cual ha ampliado la necesidad de personal con amplia experiencia, lo que resulta insuficiente para satisfacer la alta demanda de pacientes. Por lo tanto, resulta crucial el desarrollo de herramientas computacionales que permitan la localización automática, precisa y eficiente de lesiones prostáticas. En particular, resulta relevante la implementación de una herramienta basada en aprendizaje profundo que especialice la tarea de localización, para servir de soporte y mejorar el diagnóstico del cáncer de próstata clínicamente significativo csPCA en estudios de bp-MRI.



