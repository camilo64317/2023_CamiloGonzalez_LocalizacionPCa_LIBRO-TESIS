% ------------------------------------------------------------------------
% ------------------------------------------------------------------------
% ------------------------------------------------------------------------
%                                Abstract
% ------------------------------------------------------------------------
% ------------------------------------------------------------------------
% ------------------------------------------------------------------------
\chapter*{ABSTRACT}

\footnotesize{
\begin{description}
  \item[TITLE:] Localization of prostate cancer-related lesions on multimodal bp-MRI sequences. \astfootnote{Research work}
  \item[AUTHOR:] Camilo Eduardo González Guerrero \asttfootnote{Faculty of Physics-Mechanics Engineering. School of Systems Engineering and Informatics. Advisor: Fabio Martínez, Ph.D. Co-advisor: Juan Andrés Olmos Rojas, M.Sc.}
  \item[KEYWORDS:] Prostate cancer, bpMRI, csPCa lesion, multimodal
  
  \item[DESCRIPTION:] 
Prostate cancer is the second cancer with the highest incidence in men worldwide. In Colombia, for example, the death rate is around 11.6 cases per 100,000 inhabitants. Nowadays, the study of prostate lesions through biparametric magnetic resonance imaging is a standard criterion for the detection and diagnosis of prostate cancer, even in early stages. However, the localization of these lesions remains subjective and their characterization reports low levels of sensitivity. This is why computational mechanisms are nowadays key for the localization, diagnosis and triage of prostate cancer on bp-MRI studies. In this proposal we intend to develop a deep learning tool to localize prostate lesions from the assessment of multiple MRI parameters. For the above, a dataset properly annotated by experts will be selected, a deep learning tool will be adjusted and adapted. The proposed schema is expected to be validated in terms of its ability to localize lesions and discriminate them according to the stratification of the dataset.   

 


\end{description}
}\normalsize