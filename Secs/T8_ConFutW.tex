


\chapter{CONCLUSIONES Y TRABAJO FUTURO}


% En este trabajo se implementó y adaptó un \textit{autoencoder} variacional para codificar patrones geométricos que correspondan a la síntesis de nanoespumas con diferentes composiciones. El autoencoder fue ajustado, siguiendo una tarea de reconstrucción de imágenes, y logrando codificar vectores embebidos con la capacidad de discriminación entre diferentes geometrías de las nanoespumas. 

% En este trabajo, utilizando observaciones de un microscopio confocal, se lograron observar patrones codificados en los vectores embebidos para representar patrones geométricos de las nanoespumas. Este hecho constituye un potencial de las herramientas computacionales para la caracterización y cuantificación de la geometría sin dedicarse a análisis tediosos y bajo medidas simplificadas de los poros.  Estos modelos son discriminables a partir del uso de herramientas de aprendizaje de máquina como \textit{Random Forest, KNN y decistion tree}, logrando niveles de precisión por encima del 90\%. 

% Además, los resultados evidencian que se pueden simular entornos experimentales para la construcción de nanoespumas, siendo este hecho de gran potencial en los expertos del área, para validar el comportamiento geométrico, frente a diferentes composiciones de cobre-níquel. Como herramienta alternativa, también se obtuvieron mapas de explicabilidad que pueden dar evidencia de los componentes geométricos que se codifican en el vector embebido. Así, logrando enfocarse en características como el espacio entre los poros y la geometría del poro, características usadas por los expertos durante el análisis. 

% Como trabajo futuro se esperan continuar indagando en nuevas arquitecturas que tengan la capacidad de representar la distribución de características geométricas, además que tengan el carácter discriminativo entre diferentes composiciones del material. También se pretende investigar el comportamiento con observaciones en otras magnificaciones, para que las herramientas computacionales puedan suplir requerimientos experimentales que pueden llegar a ser costosos o tomar tiempos prolongados en la síntesis de las nanoespumas.
